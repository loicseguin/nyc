\PassOptionsToPackage{unicode=true}{hyperref} % options for packages loaded elsewhere
\PassOptionsToPackage{hyphens}{url}
%
\documentclass[]{article}
\usepackage{lmodern}
\usepackage{amssymb,amsmath}
\usepackage{ifxetex,ifluatex}
\usepackage{fixltx2e} % provides \textsubscript
\ifnum 0\ifxetex 1\fi\ifluatex 1\fi=0 % if pdftex
  \usepackage[T1]{fontenc}
  \usepackage[utf8]{inputenc}
  \usepackage{textcomp} % provides euro and other symbols
\else % if luatex or xelatex
  \usepackage{unicode-math}
  \defaultfontfeatures{Ligatures=TeX,Scale=MatchLowercase}
\fi
% use upquote if available, for straight quotes in verbatim environments
\IfFileExists{upquote.sty}{\usepackage{upquote}}{}
% use microtype if available
\IfFileExists{microtype.sty}{%
\usepackage[]{microtype}
\UseMicrotypeSet[protrusion]{basicmath} % disable protrusion for tt fonts
}{}
\IfFileExists{parskip.sty}{%
\usepackage{parskip}
}{% else
\setlength{\parindent}{0pt}
\setlength{\parskip}{6pt plus 2pt minus 1pt}
}
\usepackage{hyperref}
\hypersetup{
            pdfborder={0 0 0},
            breaklinks=true}
\urlstyle{same}  % don't use monospace font for urls
\setlength{\emergencystretch}{3em}  % prevent overfull lines
\providecommand{\tightlist}{%
  \setlength{\itemsep}{0pt}\setlength{\parskip}{0pt}}
\setcounter{secnumdepth}{0}
% Redefines (sub)paragraphs to behave more like sections
\ifx\paragraph\undefined\else
\let\oldparagraph\paragraph
\renewcommand{\paragraph}[1]{\oldparagraph{#1}\mbox{}}
\fi
\ifx\subparagraph\undefined\else
\let\oldsubparagraph\subparagraph
\renewcommand{\subparagraph}[1]{\oldsubparagraph{#1}\mbox{}}
\fi

% set default figure placement to htbp
\makeatletter
\def\fps@figure{htbp}
\makeatother


\date{}

\begin{document}

\hypertarget{laboratoire-1-mouvement-harmonique-simple}{%
\section*{Laboratoire 1 -- Mouvement harmonique
simple}\label{laboratoire-1-mouvement-harmonique-simple}}
\addcontentsline{toc}{section}{Laboratoire 1 -- Mouvement harmonique
simple}


\hypertarget{consignes}{%
\subsection*{Consignes}\label{consignes}}
\addcontentsline{toc}{subsection}{Consignes}

Chaque équipe doit remettre :

\begin{itemize}
\tightlist
\item
  un rapport complet papier, broché, qui inclut tous les éléments
  décrits plus bas dans l'ordre;
\item
  un fichier Excel contenant tous les calculs nécessaires effectués à
  l'aide de formules appropriées. Le fichier Excel doit être remis sur
  Léa.
\end{itemize}

La date limite de remise est le jeudi 6 septembre 2018 à 17h.

Je corrigerai le fichier Excel en remplaçant vos données par les
miennes. Les valeurs finales calculées dans votre fichier devront être
correctes.

Dans votre texte, vous n'avez pas le droit d'utiliser de listes à puces
ou à numéros. Vous devez répondre aux exigences en rédigeant un texte
continu. Toutes vos phrases doivent compter un maximum de 20 mots. Votre
rapport de laboratoire doit compter un maximum de 4 pages, incluant les
tableaux et les graphiques. Je ne corrigerai pas les phrases plus
longues que 20 mots, ni les pages au-delà de la quatrième.

Votre rapport de laboratoire doit inclure les sections suivantes, dans
l'ordre.

\hypertarget{introduction}{%
\subsection*{Introduction}\label{introduction}}
\addcontentsline{toc}{subsection}{Introduction}

L'objectif de cette section est de présenter le modèle théorique que
vous voulez vérifier expérimentalement de même que la méthode d'analyse
que vous utiliserez.

Présentez le modèle du mouvement harmonique simple pour un système
bloc-ressort (équations 6 et 8 du guide de laboratoire). Prenez soin de
bien définir toutes les variables qui apparaissent dans les équations.

Résumez la méthode d'analyse que vous utiliserez. Pour valider
l'expression de la position du bloc en fonction du temps (équation 6),
vous ferez une régression sinusoïdale. Comment obtiendrez-vous les
mesures pour faire cette régression? Pour valider l'expression de la
fréquence angulaire du système, vous comparerez les valeurs obtenues par
régression sinusoïdale au modèle de l'équation 8. Comment
obtiendrez-vous une valeur de référence pour la constante du ressort
(mesure obtenue par la méthode statique)? Comment linéariserez-vous
l'équation 8 pour vérifier le modèle?

Lorsque vous linéarisez une équation, expliquez quelles sont les
quantités qui jouent le rôle de variable indépendante et dépendante.
Quelle est l'allure du graphique attendue? Quelles sont les valeurs
attendues des paramètres (pente et ordonnée à l'origine) de la droite de
régression?

\hypertarget{ruxe9sultats}{%
\subsection*{Résultats}\label{ruxe9sultats}}
\addcontentsline{toc}{subsection}{Résultats}

Cette section comportera trois tableaux et trois graphiques. Le premier
tableau présente les valeurs de fréquence angulaire pour trois
amplitudes différentes. Ces données vous permettront de vérifier si la
fréquence angulaire du MHS est indépendante de l'amplitude. À la suite
de ce tableau, justifiez les incertitudes.

Le deuxième tableau présente les données mesurées directement
(allongement du ressort et fréquence angulaire pour différentes masses
suspendues). À la suite de ce tableau, justifiez les incertitudes.

Le troisième tableau présente les données qui seront utilisées pour
tracer les graphiques. Ce tableau doit inclure les valeurs des variables
indépendante et dépendante des modèles linéarisés.

Le premier graphique est une capture d'écran de Capstone qui montre le
mouvement du bloc et la régression sinusoïdale. Le deuxième graphique
montre la relation entre les variables qui vous permet de déduire la
valeur de référence de la constante du ressort. Le troisième graphique
présente la version linéarisée de l'équation 8. Chacun des graphiques
doit occuper un minimum d'une demi-page.

\hypertarget{conclusion}{%
\subsection*{Conclusion}\label{conclusion}}
\addcontentsline{toc}{subsection}{Conclusion}

À partir de vos résultats, expliquez si le modèle du mouvement
harmonique simple est approprié pour décrire le mouvement du système
bloc-ressort. Est-ce que l'amplitude est bien indépendante de la
fréquence angulaire? Est-ce que la position en fonction du temps du bloc
est bien représentée par une fonction sinusoïdale? Est-ce que la pente
et l'ordonnée à l'origine du graphique obtenu à partir des données
expérimentales correspondent aux valeurs attendues?

Si vos modèles ne sont pas vérifiés, expliquez les sources probables du
problème.

Dans cette expérience, le ressort n'est pas idéal et s'étire sous
l'effet d'une portion de sa masse. On aurait dû faire l'analyse avec une
masse suspendue effective qui inclut une partie de la masse du ressort
\[m_\mathrm{eff} = m + \xi M_\mathrm{ressort} \;\;\text{où}\;\; \xi \in [0, 1].\]
Sauriez-vous, à partir de vos données expérimentales, déterminer la
fraction de la masse du ressort qui aurait due être considérée dans
votre analyse?

\end{document}
